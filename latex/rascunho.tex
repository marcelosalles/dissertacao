\title{Rascunhos}

\date{\today}

\documentclass[12pt]{ufscthesis}

\begin{document}
	\maketitle
	
%	\paragraph{Outline}
%	The remainder of this article is organized as follows.
%	Section~\ref{previous work} gives account of previous work.
%	Our new and exciting results are described in Section~\ref{results}.
%	Finally, Section~\ref{conclusions} gives the conclusions.
	
	\section{Túnel de vento X Método analítico}\label{TPU}
	Na base de dados da TPU, valores de cp são disponibilizados para 25 geometrias diferentes: 13 geometrias "highrise", e 12 "lowrise".
	Através do método analítico, calculou-se Cp’s para as fachadas com essas 25 geometrias.
	Para cada fachada, calculou-se a diferença média entre os Cp’s TPU;
	A geometria com a maior média das diferenças média foi utilizada para comparação de resultados de EHF e ACH através do Energyplus.
	
	Modelo com 6 escritórios;
	Apenas 1 pavimento, com piso e cobertura adiabáticos;
	Utilizou-se a média dos Cp’s sobre a janela;
	Uma amostra de 1000 casos foi gerada por LHS;
	Para cada caso, foi rodado um modelo com Cp automático do e+ (analítico), e um do TPU (túnel de vento)
	
	\section{EPS+Concrete}
	
	\section{Adiabatic X Outdoors}
	
	\section{Crack values}
	
	\section{Cp equivalente}
		
\end{document}