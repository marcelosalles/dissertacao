\documentclass[brazil,hardcopy,openany,a4paper]{ufscthesis}
\usepackage[brazil]{babel}
\usepackage{amsfonts, amsmath, amsthm, amsbsy,amssymb,bm,mathtools} % For math fonts, symbols and environments %
\usepackage{graphicx} 		% Required for including images
\usepackage{transparent}	% may be required for inkscape pdf figures (http://bit.ly/18i5Oga)
\usepackage{listings}
\usepackage[abnt-emphasize=bf]{abntex2cite}
\usepackage{caption}
\usepackage{multirow}
\usepackage{lscape}
\usepackage[T1]{fontenc}
\sloppy
\usepackage{siunitx}
\usepackage{nameref}
\usepackage{float}
\usepackage{subfig}

\newcommand{\source}[1]{\small \caption*{Fonte: {#1}} } % Criar fonte embaixo da figura

\newsubfloat{figure}		% Allow subfloats in figure environment (http://bit.ly/1C20NAj)
\graphicspath{{figures/}} 	% Location of the graphics files

\usepackage{siunitx} % units package
\let\DeclareUSUnit\DeclareSIUnit
\let\US\SI
\let\us\si
\DeclareUSUnit\inch{in}
\sisetup{detect-all}  %it may be necessary to load it after loading the font package

\citebrackets[]

%----------------------------------------------------------------------
% Comandos criados pelo usuário
\newcommand{\afazer}[1]{{\color{red}{#1}}} % Para destacar uma parte a ser trabalhada
\DeclareMathOperator*{\argmin}{\arg\!\min}
\DeclareMathOperator*{\argmax}{\arg\!\max}

%----------------------------------------------------------------------
% Identificadores do trabalho
% Usados para preencher os elementos pré-textuais
\instituicao[a]{Universidade Federal de Santa Catarina} % Opcional
\departamento[a]{Biblioteca Universitária}
\programa[o]{Programa de Pós-Graduação em Engenharia Civil} 
\curso{Engenharia de Engenharia Civil}
\documento[a]{Dissertação} % [o] para dissertação e trabalho de conclusão de curso [a] para tese
\grau{Mestre} % doutor, mestre, engenheiro, etc.
\titulo{PREDIÇÃO DE CONFORTO TÉRMICO EM ESCRITÓRIOS VENTILADOS NATURALMENTE POR MEIO DE REDES NEURAIS ARTIFICIAIS}
\subtitulo{} % Opcional
\autor{Marcelo Salles Olinger}
\local{Florianópolis} % Opcional (Florianópolis é o padrão)
\data{28}{Fevereiro}{2019}
\orientador[Universidade Federal de Santa Catarina]{Profa. Ana Paula Melo, Dra.}

\begin{document}
	
	\frontmatter
	\folhaderosto[]%pre/Ficha_Catalografica.pdf]
	
	\mainmatter

	\chapter{Resultados}
	\label{chapter:Resultados}
	
	\section{Parâmetros de entrada}
	Ao analisar o banco de dados disponibilizado, obteve-se as distribuições de ocorrência em relação aos parâmetros observados (Figura \ref{fig:db_hist}).
		
	\begin{figure}[H]
		\captionof{figure}{Distribuições de ocorrência}
		\label{fig:db_hist}
		\centering
		\begin{minipage}{.5\textwidth}
			\centering
			\includegraphics[width=\linewidth]{img/hist_azimute.png}
		\end{minipage}%
		\begin{minipage}{.5\textwidth}
			\centering
			\includegraphics[width=\linewidth]{img/hist_numero_pavimentos.png}
		\end{minipage}
		\centering
		\begin{minipage}{.5\textwidth}
			\centering
			\includegraphics[width=\linewidth]{img/hist_formato.png}
		\end{minipage}%
		\begin{minipage}{.5\textwidth}
			\centering
			\includegraphics[width=\linewidth]{img/hist_formato_sala.png}
		\end{minipage}
		\centering
		\begin{minipage}{.5\textwidth}
			\centering
			\includegraphics[width=\linewidth]{img/hist_pe_direito.png}
		\end{minipage}%
		\begin{minipage}{.5\textwidth}
			\centering
			\includegraphics[width=\linewidth]{img/hist_area_zonas.png}
		\end{minipage}
		\centering
		\begin{minipage}{.5\textwidth}
			\centering
			\includegraphics[width=\linewidth]{img/hist_ratio_edificio.png}
		\end{minipage}%
		\begin{minipage}{.5\textwidth}
			\centering
			\includegraphics[width=\linewidth]{img/hist_ratio_sala.png}
		\end{minipage}
	\end{figure}
	\begin{figure}
		\ContinuedFloat
		\caption[]{\textit{Continuação}} % {figure}
		\centering	
		\begin{minipage}{.5\textwidth}
			\centering
			\includegraphics[width=\linewidth]{img/hist_absortancia.png}
		\end{minipage}%
		\begin{minipage}{.5\textwidth}
			\centering
			\includegraphics[width=\linewidth]{img/hist_cor_cobertura.png}
		\end{minipage}
		\centering	
		\begin{minipage}{.5\textwidth}
			\centering
			\includegraphics[width=\linewidth]{img/hist_tipo_vidro.png}
		\end{minipage}%
		\begin{minipage}{.5\textwidth}
			\centering
			\includegraphics[width=\linewidth]{img/hist_esquadria.png}
		\end{minipage}
		\centering	
		\begin{minipage}{.5\textwidth}
			\centering
			\includegraphics[width=\linewidth]{img/hist_openfac.png}
		\end{minipage}%
		\begin{minipage}{.5\textwidth}
			\centering
			\includegraphics[width=\linewidth]{img/hist_PAF.png}
		\end{minipage}
		\centering	
		\begin{minipage}{.5\textwidth}
			\centering
			\includegraphics[width=\linewidth]{img/hist_sombreamento.png}
		\end{minipage}%
		\begin{minipage}{.5\textwidth}
			\centering
			\includegraphics[width=\linewidth]{img/hist_tipo_vn.png}
		\end{minipage}
	\end{figure}
	
	Tanto os edifícios, quanto as salas existentes no banco de dados apresentam predominantemente formato retangular, a partir do qual considera-se que definir as simulações baseando-se em modelos de edificações retangulares, com salas retangulares, representa adequadamente as tipologias de edifícios encontradas na cidade de São Paulo.
	A absortância da cobertura foi definida com o valor fixo de 0,7, valor aproximado para uma cobertura de cor cinza.
	
	Observou-se que esquadrias do tipo maxim-ar são predominantes. Os objetos do \textit{Airflow Network} não modelam especificamente este tipo de esquadria. Porém, optou-se por considerar as janelas como não pivotantes. Considerar uma janela como horizontalmente pivotante implicaria na consideração de que a abertura acontece simultaneamente em cima e embaixo da janela. No caso da janela maxim-ar, por mais que a abertura aconteça em um eixo horizontal, apenas a parte inferior da janela abre.
	
	O uso de elementos de sombreamento é pouco explorado nas edificações existentes. De qualquer maneira, considerou-se a modelagem de sombreamento horizontal sobre as aberturas da edificação, por considerar o potencial do sombreamento para bloquear a entrada de radiação nas zonas térmicas simuladas. Esse parâmetro foi variado a partir do ângulo de sombreamento formado entre a base da abertura e a proteção solar, localizada no topo da abertura.	
	A maioria das salas observadas possuem ventilação cruzada, mas a ventilação unilateral é uma estratégia com ocorrência considerável.
	
	As informações relacionadas ao tipo de vidro não permitem definir valores relacionados ao fator solar (FS). Observa-se apenas a ocorrência de vidros laminados e vidro comum incolor. Optou-se por variar o fator solar dos vidros nas simulações para avaliar o impacto deste parâmetro nos resultados de conforto térmico.
	
%	Os demais parâmetros observados tiveram suas distribuições variando continuamente de acordo com as distribuições obtidas. 
	Como as simulações de referência foram modeladas como pavimentos da edificação, o parâmetro relacionado ao número de pavimento das edificações foi transformado no parâmetro "altura do pavimento".
	
	A Tabela \ref{table:param_def} apresenta os limites mínimos e máximos atribuídos aos diferentes parâmetros contínuos variados nas simulações, assim como os parâmetros variados pela lógica "sim/não". A velocidade do ar foi variada com valores discretos, de acordo com a Tabela \ref{table:var} do Capítulo \ref{chapter:metodologia}.
	
		\begin{table}[H]
			\centering
			\caption{Limites mínimos e máximos dos parâmetros}
			\label{table:param_def}
			\begin{tabular}{|l |r |}
				\hline
				\textbf{Parâmetro} & \textbf{Valores} \\
				\hline
				Área da sala (m$^2$) & 20 - 100 \\
				\hline
				Razão L:C da sala (-) & 0,4 - 2,5 \\
				\hline
				Pé-direito (m) & 2,3 - 3,2 \\
				\hline
				Azimute ($^{\circ}$) & 0 - 360 \\
				\hline
				Altura do pavimento (m) & 0 - 50 \\
				\hline 
				Absortância da parede (-) & 0,2 - 0,8 \\
				\hline 
				Transmitância da parede (W/m$^2$K) & 0,5 - 4,4 \\
				\hline 
				Capacidade térmica da parede (kJ/m$^2$K) & 0,22 - 450,00 \\
				\hline 
				PAF (-) & 0,1 - 0,6 \\
				\hline 
				Fator solar do vidro (-) & 0,20 - 0,87 \\
				\hline 
				Sombreamento ($^{\circ}$) & 0 - 80 \\
				\hline 
				Densidade de ocupação (pessoa/m$^2$) & 0,05 - 0,20 \\
				\hline 
				Fator de abertura da janela (-) & 0,2 - 1,0 \\
				\hline 
				Razão L:C do edifício (-) & 0,2 - 1,0 \\
				\hline 
				Cobertura exposta & Sim / Não\\
				\hline 
				Piso exposto & Sim / Não\\
				\hline 
				Ventilação & Cruzada / Unilateral\\
				\hline 
				Velocidade do ar (m/s) & 0,0 - 1,2 \\
				\hline 
			\end{tabular}
%			\begin{flushleft}
%				Fonte: \citeauthoronline{INIC} \cite{INIC}, adaptado pelo autor.
%			\end{flushleft}				
		\end{table}
		
		\section{Simulações simplificadas}
		
		\subsection{Cálculo do coeficiente de pressão pelo método analítico}
		
		Ao comparar os valores dos coeficientes de pressão (Cp’s) das medições em túnel de vento da Universidade Politécnica de Tóquio (TPU) e os valores dos Cp’s obtidos pelo método analítico (MA), obteve-se um gráficos de pontos. 
		A Figura \ref{fig:cp_diff_scatter_all} apresenta a comparação para as 25 geometrias diferentes, para cada fachada, e para cada ponto disponibilizado pela TPU.
		Como os valores calculados pelo MA são únicos para cada fachada, e a TPU oferece valores diferentes para diversos pontos ao longo das fachadas, os pontos no gráfico da Figura \ref{fig:cp_diff_scatter_all} distribuem-se horizontalmente. 
%		A Figura \ref{fig:cp_diff_scatter_facade} apresenta a comparação considerando-se os valores médios do Cp para cada fachada. 
		É possível observar que a faixa de valores dos Cp's disponibilizados pela TPU é maior do que  faixa de valores calculados pelo MA. Enquanto o menor valor de Cp disponibilizado pela TPU é -1,40, e o maior valor é 1,08, pelo MA o valor mínimo é igual a -0,96 e o máximo é igual a 0,60.
		
		\begin{figure}[H]
			\centering
			\caption{Comparação entre os valores de Cp das 25 geometrias}
			\includegraphics[width=1\linewidth]{img/cp_diff_scatter_all.png}
			\label{fig:cp_diff_scatter_all}
			%			\begin{flushleft}
			%				Fonte: o autor.
			%			\end{flushleft}
		\end{figure}
		
		Dentre as geometrias analisadas, a proporção com a maior diferença absoluta média entre os valores dos Cp’s foi igual a 0,344, para a geometria da edificação \textit{highrise} com proporções de largura, profundidade e altura igual a 2:1:2 (Figura \ref{fig:cp_diff_scatter}).
	
		\begin{figure}[H]
			\centering
			\caption{Comparação entre os valores de Cp da geometria de proporções 2:1:2}
			\includegraphics[width=1\linewidth]{img/cp_diff_scatter.png}
			\label{fig:cp_diff_scatter}
			%			\begin{flushleft}
			%				Fonte: o autor.
			%			\end{flushleft}
		\end{figure}
	
		A partir da comparação conduzida, decidiu-se comparar as diferenças nos resultados de simulações termoenergéticas utilizando como base uma tipologia com proporções de largura, profundidade e altura igual a 2:1:2.
		
		Os resultados das 1000 simulações foram comparados por gráficos de pontos.
		A comparação entre as médias anuais de trocas de ar por hora (ACH), apresentada na Figura \ref{fig:cpaverage_ACH_scatter}, mostra que o MA faz com que as simulações subestimem as trocas de ar nas simulações, possivelmente devido aos menores valores dos Cp's obtidos pelo método.
		A diferença média foi igual a -0,67 para o ACH, com o erro absoluto do 95º percentil (AE95) é igual a 5,23 ACH.
		
		\begin{figure}[H]
			\centering
			\caption{Comparação entre as médias anuais do ACH}
			\includegraphics[width=1\linewidth]{img/cpaverage_ACH_scatter.png}
			\label{fig:cpaverage_ACH_scatter}
			%			\begin{flushleft}
			%				Fonte: o autor.
			%			\end{flushleft}
		\end{figure}
		
		Apesar dessas diferenças nas trocas de ar, a comparação entre as temperaturas operativas médias, apresentada na Figura \ref{fig:cpaverage_temp_scatter}, mostra que a diferença média da temperatura operativa é 0,04 $^{\circ}$C, sendo que o AE95 é igual a 0,31 $^{\circ}$C.
		Essas diferenças são confirmadas como pouco significativas ao se analisar a Figura \ref{fig:cpaverage_EHF_scatter}, com a comparação da fração de horas em desconforto (EHF). A média de diferença do EHF nos casos analisados foi igual a 0,0037, com o AE95 igual a 0,0277.
		Por tanto, considerou-se que a utilização do MA para calcular os valores dos Cp's é uma alternativa adequada para a simplificação das simulações termoenergéticas.
		
		\begin{figure}[H]
			\centering
			\caption{Comparação entre as médias das temperaturas operativas}
			\includegraphics[width=1\linewidth]{img/cpaverage_temp_scatter.png}
			\label{fig:cpaverage_temp_scatter}
			%			\begin{flushleft}
			%				Fonte: o autor.
			%			\end{flushleft}
		\end{figure}
	
		\begin{figure}[H]
			\centering
			\caption{Comparação entre os resultados de EHF}
			\includegraphics[width=1\linewidth]{img/cpaverage_EHF_scatter.png}
			\label{fig:cpaverage_EHF_scatter}
			%			\begin{flushleft}
			%				Fonte: o autor.
			%			\end{flushleft}
		\end{figure}
	
	\subsection{Representação da envoltória com duas camadas}
		
		Os resultados das simulações com as paredes equivalentes subestimaram o EHF em 0,0107 na média, quando comparados aos resultados das simulações com as paredes de referência. 
		Os resultados das simulações para a parede de gesso com isolamento resultaram em um erro absolto médio igual a 0,0100, e um AE95 igual a 0,0604 para o EHF (Figura \ref{fig:par3_scatter}).
		
		\begin{figure}[H]
			\centering
			\caption{Comparação entre os resultados de EHF para a parede de gesso com isolamento}
			\includegraphics[width=1\linewidth]{img/paredeeq_EHF_par3_scatter.png}
			\label{fig:par3_scatter}
			%			\begin{flushleft}
			%				Fonte: o autor.
			%			\end{flushleft}
		\end{figure}
		
		A representação da parede de alvenaria apresenta-se mais adequada considerando-se apenas metade da parede para definir a capacidade térmica. Enquanto que, para a parede com a capacidade térmica total o erro absoluto médio foi igual a 0,0209, e o AE95 foi igual a 0,0650, para a parede de alvenaria com metade da capacidade térmica considerada, o erro médio absoluto foi igual a 0,0189, e o AE95 foi igual a 0,0604.
		
		\begin{figure}[H]
			\centering
			\caption{Comparação entre os resultados de EHF para a parede de alvenaria com capacidade térmica total}
			\includegraphics[width=1\linewidth]{img/paredeeq_EHF_par2a_scatter.png}
			\label{fig:par2a_scatter}
			%			\begin{flushleft}
			%				Fonte: o autor.
			%			\end{flushleft}
		\end{figure}
		\begin{figure}[H]
			\centering
			\caption{Comparação entre os resultados de EHF para a parede de alvenaria com metade da capacidade térmica}
			\includegraphics[width=1\linewidth]{img/paredeeq_EHF_par2b_scatter.png}
			\label{fig:par2b_scatter}
			%			\begin{flushleft}
			%				Fonte: o autor.
			%			\end{flushleft}
		\end{figure}
		
		O caso com as maiores diferenças no EHF foi para uma edificação em contato com o solo, com cobertura exposta, e um fator de abertura da janela igual a 0,23.
		Apesar das diferenças nos resultados, o uso da parede equivalente facilita a parametrização da transmitância térmica e da capacidade térmica. Por esse motivo, considerou-se as diferenças pouco significativas, e a parede equivalente foi adotada para simplificar as simulações.
		
		\subsection{Condição de contorno das paredes adjacentes à edificação}
		
		A simplificação das simulações adotando-se apenas uma zona térmica foi avaliada para duas condições de contorno. Os resultados mostram que a maneira mais adequada de representar as paredes adjacentes à circulação da edificação é considerando-as como adiabáticas. Considerar as paredes adjacentes à circulação como \textit{Outdoors}, faz com que os resultados do EHF sejam subestimados em 0,0868 em média, como AE95 igual a 0,1865 (Figura \ref{fig:szout_EHF}).
		
		\begin{figure}[H]
			\centering
			\caption{Comparação entre os resultados de EHF para parede \textit{Outdoors}}
			\includegraphics[width=1\linewidth]{img/szout_EHF_scatter.png}
			\label{fig:szout_EHF}
			%			\begin{flushleft}
			%				Fonte: o autor.
			%			\end{flushleft}
		\end{figure}
		
		Os resultados das simulações considerando-se as paredes voltadas para o corredor como adiabáticas subestimaram o EHF em 0,0051 na média, como AE95 igual a 0,0804 (Figura \ref{fig:szadi_EHF}).
		
		\begin{figure}[H]
			\centering
			\caption{Comparação entre os resultados de EHF para parede adiabática}
			\includegraphics[width=1\linewidth]{img/szadi_EHF_scatter.png}
			\label{fig:szadi_EHF}
			%			\begin{flushleft}
			%				Fonte: o autor.
			%			\end{flushleft}
		\end{figure}
		
		A partir dos resultados obtidos, definiu-se as paredes voltadas para a circulação como adiabáticas no desenvolvimento das simulações simplificadas.
		
	\subsection{Modelagem da ventilação natural na simulação simplificada}
	
	Nesta etapa do trabalho as simulações foram conduzidas para se obter duas respostas:
	(1) se é adequado o uso do coeficiente de pressão equivalente ($Cp_{eq}$) para ser associado à porta da zona térmica; (2) qual deveria ser o coeficiente de fluxo mássico de ar adotado para o objeto \textit{AirflowNetwork:MultiZone:Surface:Crack}.
	
	Para analisar simultaneamente o desempenho do $Cp_{eq}$ e dos coeficientes de fluxo mássico de ar, o gráfico da Figura \ref{fig:pareto} foi gerado, observando-se as raízes dos erros médios quadráticos (RMSE).
	É possível observar que as simulações desenvolvidas utilizando-se o $Cp_{eq}$ obtiveram resultados com RMSE menores do que as simulações desenvolvidas utilizando-se o Cp obtido diretamente pelo MA.
	Para a definir o coeficiente de fluxo mássico de ar, levou-se em conta inicialmente os erros relacionados ao ACH.
	No entanto, foi identificada uma fronteira de Pareto entre os erros analisados, que mostra como a busca por menores erros de ACH aumenta os erros relacionados ao EHF.
	O resultado dessa análise levanta duas suspeitas. A primeira é de que as diferenças maiores nas trocas de ar anulem erros relacionados à definição das paredes adjacentes à edificação como adiabáticas. A segunda hipótese, é de que os maiores erros relacionados ao MACH sejam em casos onde as diferenças nas trocas de ar não sejam relevantes para alterar a temperatura operativa nas zonas térmicas e, consequentemente, o EHF.
	
	\begin{figure}[H]
		\centering
		\caption{Eficiência de Pareto entre EHF e ACH médio}
		\includegraphics[width=1\linewidth]{img/cpeq_pareto.png}
		\label{fig:pareto}
		%			\begin{flushleft}
		%				Fonte: o autor.
		%			\end{flushleft}
	\end{figure}
	
	Como o desenvolvimento das simulações é voltado para obter a maior precisão possível para os resultados de EHF, optou-se por definir o coeficiente de fluxo mássico de ar com valor igual a 0,8, pois as simulações desenvolvidas utilizando-se este valor estão na fronteira de Pareto, e resultaram nos menores erros de EHF.
	
	\subsection{Análise de sensibilidade}
	
	As Figuras \ref{fig:as_ach}, \ref{fig:as_temp} e \ref{fig:as_ehf} apresentam os resultados das análises de sensibilidade (AS) para efeitos de primeira ordem e efeitos totais, relacionados ao ACH, às temperaturas operativas das zonas, e ao EHF. Os índices apresentados são proporcionais às influências entre os dados de entrada e saída.
	
	\begin{figure}[H]
		\centering
		\caption{AS de Sobol dos efeitos de primeira ordem e efeitos totais nas médias anuais de ACH}
		\includegraphics[width=1\linewidth]{img/as_ach.png}
		\label{fig:as_ach}
		%			\begin{flushleft}
		%				Fonte: o autor.
		%			\end{flushleft}
	\end{figure}
	
	\begin{figure}[H]
		\centering
		\caption{AS de Sobol dos efeitos de primeira ordem e efeitos totais nas temperaturas operativas}
		\includegraphics[width=1\linewidth]{img/as_temp.png}
		\label{fig:as_temp}
		%			\begin{flushleft}
		%				Fonte: o autor.
		%			\end{flushleft}
	\end{figure}
	
	\begin{figure}[H]
		\centering
		\caption{AS de Sobol dos efeitos de primeira ordem e efeitos totais no EHF}
		\includegraphics[width=1\linewidth]{img/as_ehf.png}
		\label{fig:as_ehf}
		%			\begin{flushleft}
		%				Fonte: o autor.
		%			\end{flushleft}
	\end{figure}

	Os parâmetros mais influentes nas ACH, como esperado, são aqueles relacionados às aberturas da zona. 
	O primeiro parâmetro de maior influência é o fator de abertura das janelas, seguido do parâmetro relacionado à exposição das paredes e à presença de VN cruzada ou unilateral. A área tem influência significativa, pois o cálculo das trocas de ar leva em conta o volume de ar na zona, que é diretamente relacionado à sua área. 
	A altura do pavimento é determinante nos resultados do ACH, pois a velocidade do vento no EnergyPlus é calculada em função da altura da zona.
	A orientação da zona (azimute) não tem uma influência significativa de primeira ordem. No entanto, percebe-se uma influência mais significativa considerando-se os efeitos totais. 
	O azimute é determinante para a definição dos coeficientes de pressão sobre as fachadas da edificação. Por isso, a influência deste parâmetro nos resultados das simulações depende de outros parâmetros, relacionados ao posicionamento e às áreas das aberturas na zona.
	A velocidade do ar não influencia os resultados relacionados ao ACH, pois é considerada apenas após o término das simulações, ao se calcular o EHF.
	A AS apresentou interações de segunda ordem significativas entre o fator de abertura das janelas e a presença de VN cruzada ou unilateral, com um índice de sensibilidade igual a 0,121.
	Contudo, o parâmetro com maiores interações de segunda ordem relacionados ao ACH foi o PAF, com a soma dos índices de segunda ordem igual a 0,300.
	
	As análises relacionadas à temperatura operativa e ao EHF indicam relevância dos parâmetros relacionados à VN. Para ambas as análises, o parâmetro mais influente foi o fator de abertura da janela, enquanto o parâmetro relacionado à exposição das paredes e à presença de VN cruzada ou unilateral foi o terceiro mais influente. 
	O contato com o solo apresentou-se como o segundo parâmetro mais influente nas médias anuais de temperatura operativa, considerando-se os esfeitos totais. No entanto, a influência deste parâmetro não é tão significativa no EHF. Isso indica que a influência do contato com o solo nas temperaturas operativas das zonas é mais significativa em faixas de temperatura que não interferem no cálculo do EHF, ou seja, consideravelmente a cima ou abaixo dos limites superiores de aceitabilidade estabelecidos pelo método de conforto adaptativo.
	Observa-se que os efeitos totais entre o segundo (contato com o solo) e o quarto (exposição da cobertura) índice de sensibilidade com valores mais altos na AS relacionada à média anual da temperatura operativa são expressivos.
	A transmitância das paredes, o azimute, e a razão entre a largura e o comprimento da sala também apresentam efeitos totais relevantes, apesar dos baixos índices de sensibilidade para primeira ordem. Isso indica que há interações significativas entre esses parâmetros e os demais.
		
	O movimento do ar apresenta-se como o segundo parâmetro mais influente nos resultados de EHF, o que indica um grande potencial de uso de ventiladores na busca por conforto térmico nos ambientes. 
	A área da zona e a densidade de ocupação apresentaram-se mais influentes nos resultados de EHF, comparando-se aos resultados relacionados às médias anuais de temperatura operativa.
	O azimute, apesar de seu índice de sensibilidade baixo para a análise de primeira ordem, apresentou índices de segunda ordem expressivos. As interações de segunda ordem ocorrem relacionadas a parâmetros referentes à VN e a parâmetros referentes à radiação solar. A soma dos índices de segunda ordem do azimute em relação ao EHF foi igual a 0,177.  %, como sombreamento e absortância
	
	A complexidade dos fenômenos representados junto às interações entre as diferentes variáveis exige um grande número de casos para reduzir incertezas, pois o método de AS utiliza uma base amostral. Por isso, existe uma incerteza associada aos índices de sensibilidade obtidos nas AS conduzidas, e a soma dos valores dos índices ultrapassa o valor 1. Entretanto, a aplicação da análise de sensibilidade global ofereceu resultados relevantes para o trabalho, com índices de sensibilidade condizentes aos comportamentos físicos representados pelas simulações. 
	
	Baseando-se nos resultados das AS, alguns dos parâmetros não foram considerados para o desenvolvimento do metamodelo. A Tabela \ref{table:param_fixed} apresentam os parâmetros que tiveram seus valores fixos. O valor do pé-direito foi determinado considerando-se o valor encontrado com mais frequência na base de dados analisada. A capacidade térmica da parede foi estabelecida de acordo com o valor de uma parede de bloco cerâmico de dimensões 14x19x29 cm, e argamassa de 2,5 cm. Os parâmetros relacionados às proporções entre largura e profundidade das salas e edifícios foram determinados com valor igual a 1. 
	
	\begin{table}[H]		
		\centering
		\caption{Parâmetros com valores constantes.}
		\label{table:param_fixed}
		\begin{tabular}{|l |c |}
			\hline
			\textbf{Parâmetro} & Valor fixo \\
			\hline
			Razão L:C do edifício (-) & 1 \\
			\hline
			Razão L:C da sala (-) & 1 \\
			\hline
			Pé-direito (m) & 2,5 \\
			\hline
			Capacidade térmica (kJ/m$^2$K) & 161 \\
			\hline
%			Fator solar do vidro (-) & 0,87 \\
%			\hline
		\end{tabular}
	\end{table}
	
%	Occupant density is the most influential parameter
%	on EHF, since it obtained the highest index value for
%	first order, second order and total effects. From the
%	first order effects indices, it was observed that, after
%	occupant density, the contact with the ground and
%	the window opening factor are the most influential
%	parameters on both EHF and average Operative Tem-
%	perature. The index values for these two outputs are
%	similar overall, since EHF is calculated from the Op-
%	erative Temperature. Therefore, the annual average
%	Operative Temperature results were not plotted. The
%	most influential parameters on the ACH are, as ex-
%	pected, the ones related to the zone’s openings. They
%	are: opening factor of the window; exposure condition
%	of walls and windows; and WWR.
%	Significant colinearities are identified among the pa-
%	rameters. Even though the zone area has a higher
%	first order index, the exposure of the walls and the
%	roof could be more influential on the EHF, when com-
%	bined with certain parameters. Second order effects
%	analysis shows high interaction between the exterior
%	walls exposure and the opening factor of the win-
%	dow. The interaction could be explained due to the
%	the higher influence of either one or two windows on
%	the facade when the window opening factor is higher.
%	The highest second order effect index for EHF was
%	the one relating occupant density and ground expo-
%	sure. These two parameters were also the two most
%	influential for first order and total effects analyses.
%	
%	Thus, this strong relation points out how contact with
%	the ground can help to dissipate internal heat gains
%	at the climate of Sao Paulo.
%	Based on the SA results, some the least influential
%	parameters were not considered as input features for
%	the final model. They are: height of the zone’s floor;
%	SHGC; wall thermal capacity; shading device; build-
%	ing ratio.

	\subsection{Desenvolvimento do metamodelo}
	
	A partir das 20.000 simulações geradas para o treinamento da rede neural artificial (RNA). %, obteve-se resultados de EHF variando entre 0,00 e 1,00.
	O metamodelo final foi definido com 14 parâmetros:
	\begin{itemize}
		\item Fator de abertura das janelas;
		\item Velocidade do ar;
		\item Condição de exposição das paredes e janelas;
		\item Área da sala;
		\item Densidade de ocupação;
		\item Altura do pavimento;
		\item Exposição da cobertura;
		\item Sombreamento horizontal;
		\item Contato com o solo;
		\item Transmitância das paredes;
		\item Absortância das paredes;
		\item Fator solar do vidro;
		\item Azimute da sala;
		\item PAF.
	\end{itemize}
	
	Os parâmetros variaram na mesma faixa de valores estabelecida na primeira etapa deste estudo. O ângulo do azimute da sala é determinado considerando-se o eixo entre a parede voltada para a circulação e a parede oposta.
	
	O contato com o solo e a exposição da cobertura foram definidas como variáveis binárias, com o valor zero correspondendo à superfície adiabática, e 1 correspondendo à exposição.
	O parâmetro que representa a condição de exposição das paredes e janelas não foi representado com valores numéricos, e sim como uma variável de fatores, com cinco opções de exposição. Além das três opções apresentadas na Figura \ref{fig:exp_sz}, considerou-se as exposições espelhadas. 	
	Os demais parâmetros foram normalizados com valores entre -1 e 1.
	
	\begin{figure}[H]
		\centering
		\caption{Condição de exposição das paredes e janelas.}
		\includegraphics[width=1\linewidth]{img/wallexposition.png}
		\label{fig:exp_sz}
		%			\begin{flushleft}
		%				Fonte: o autor.
		%			\end{flushleft}
	\end{figure}
	
	O modelo de RNA final foi definido com duas camadas, umas de 50 nós, e a outra com 20. 
	O algorítimo de otimização que obteve o melhor desempenho foi o \textit{Adagrad's Optimizer}, disponibilizado pela biblioteca \textit{TensorFlow}, com uma taxa de aprendizagem igual a 0,05.
	
	A Figura \ref{fig:ann_validation} apresenta um gráfico de pontos comparando os resultados de EHF obtidos para as simulações e para as estimativas da RNA, a partir da base de dados desenvolvida para a validação do metamodelo. A base de dados para a validação teve apenas os parâmetros incluídos no treinamento da RNA variados. 
	O erro absoluto médio do EHF para os casos de validação foi 0,0104, com o AE95 igual a 0,0226.
	
	\begin{figure}[H]
		\centering
		\caption{Condição de exposição das paredes e janelas.}
		\includegraphics[width=1\linewidth]{img/ann_validation.png}
		\label{fig:ann_validation}
	\end{figure}
	
	Outra comparação foi conduzida com a amostragem utilizada para a AS. Essa base de dados estava disponível, e não foi utilizada pra o desenvolvimento da RNA, então ela foi escolhida para testar o desempenho da RNA quando todos os parâmetros avaliados neste estudo variam. A Figura \ref{fig:ann_sobol} apresenta o gráfico de pontos comparando os resultados de EHF obtidos para as simulações e para as estimativas da RNA, a partir da base de dados da AS de sobol. O erro absoluto médio do EHF para os casos de validação foi 0,0104, com o AE95 igual a 0,0226.
	
	\begin{figure}[H]
		\centering
		\caption{Condição de exposição das paredes e janelas.}
		\includegraphics[width=1\linewidth]{img/ann_test.png}
		\label{fig:ann_sobol}
	\end{figure}
	
	

\bibliography{citacoes}
	
\end{document}