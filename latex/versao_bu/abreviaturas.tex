%\gls{q} \Gls{latex} \Glspl{formula} \acrshort{gcd} \acrlong{DCC} \acrfull{lcm}
% Abreviaturas
\newacronym{vn}{VN}{ventilação natural}
\newacronym{iea}{IEA}{Agência Internacional de Energia}
\newacronym{aivc}{AIVC}{\textit{Air Infiltration and Ventilation Centre}}
\newacronym{hvac}{HVAC}{\textit{heating, ventilating and air conditioning}}
\newacronym{pmv}{PMV}{voto predito médio}
\newacronym{ann}{ANN}{redes neurais artificiais}
\newacronym{svm}{MVS}{máquinas de vetores de suporte}
\newacronym{cfd}{CFD}{\textit{Computer Fluid Dynamics}}
\newacronym{afn}{AFN}{\textit{Airflow Network}}
\newacronym{ct}{CT}{capacidade térmica}
\newacronym{cp}{$C_p$}{coeficientes de pressão}
\newacronym{cq}{$C_Q$}{coeficiente de fluxo mássico de ar}
\newacronym{cpeq}{$C_{p,eq}$}{coeficiente de pressão equivalente}
\newacronym{cd}{$C_d$}{coeficiente de descarga}
\newacronym{tpu}{TPU}{Universidade Politécnica de Tóquio}
\newacronym{paf}{PAF}{percentual de abertura na fachada}
\newacronym{fs}{FS}{fator solar}
\newacronym{as}{AS}{análise de sensibilidade}
\newacronym{ehf}{EHF}{fração de horas de desconforto por calor}
\newacronym{ach}{ACH}{trocas de ar por hora}
\newacronym{rmse}{RMSE}{raiz quadrada do erro quadrático médio}
\newacronym{nrmse}{NRMSE}{raiz quadrada do erro quadrático médio normalizada}
\newacronym{r2}{R$^2$}{coeficiente de determinação}
\newacronym{abse}{MAE}{erro absoluto médio}
\newacronym{ae95}{AE95}{erro absoluto do 95º percentil}
\newacronym{ma}{MA}{método analítico}
\newacronym{inic}{INI-C}{Proposta de Instrução Normativa do Inmetro para a Classe de Eficiência Energética de Edificações Comerciais, de Serviços e Públicas}

%List of Symbols
\newglossaryentry{px}{	% how the symbol will be called in the text \gls{x}
type=simbolos,  % symbols,		% set the glossary entry type as "symbol"
name={\ensuremath{P_x}},	
description={Pressão estática do ar no ambiente externo},
user1=\unexpanded{\si{\pascal}}
}
\newglossaryentry{p0}{	% how the symbol will be called in the text \gls{x}
type=simbolos,		% set the glossary entry type as "symbol"
name={\ensuremath{P_{\infty}}},	
description={Pressão estática de referência do ar},
user1=\unexpanded{\si{\pascal}}
}
\newglossaryentry{pd}{	% how the symbol will be called in the text \gls{x}
type=simbolos,		% set the glossary entry type as "symbol"
name={\ensuremath{P_d}},	
description={Pressão dinâmica do ar},
user1=\unexpanded{\si{\pascal}}
}
\newglossaryentry{rho}{	% how the symbol will be called in the text \gls{x}
type=simbolos,		% set the glossary entry type as "symbol"
name={\ensuremath{\rho}},	
description={Densidade do ar},
user1=\unexpanded{\si{\kilogram/\cubic\meter}}
}
\newglossaryentry{vref}{	% how the symbol will be called in the text \gls{x}
type=simbolos,		% set the glossary entry type as "symbol"
name={\ensuremath{V_{\infty}}},	
description={Velocidade do vento no ambiente externo},
user1=\unexpanded{\si{\meter/\second}}
}
\newglossaryentry{tinf}{	% how the symbol will be called in the text \gls{x}
type=simbolos,		% set the glossary entry type as "symbol"
name={\ensuremath{T_{inf}}},	
description={Limite inferior da temperatura operativa para 80\% de aceitabilidade no conforto térmico},
user1=\unexpanded{\si{\celsius}}
}
\newglossaryentry{tsup}{	% how the symbol will be called in the text \gls{x}
type=simbolos,		% set the glossary entry type as "symbol"
name={\ensuremath{T_{sup}}},	
description={Limite superior da temperatura operativa para 80\% de aceitabilidade no conforto térmico},
user1=\unexpanded{\si{\celsius}}
}
\newglossaryentry{tm}{	% how the symbol will be called in the text \gls{x}
type=simbolos,		% set the glossary entry type as "symbol"
name={\ensuremath{T_m}},	
description={Temperatura média do ar externo},
user1=\unexpanded{\si{\celsius}}
}
\newglossaryentry{tssup}{	% how the symbol will be called in the text \gls{x}
type=simbolos,		% set the glossary entry type as "symbol"
name={\ensuremath{timesteps_{sup}}},	
description={Número de \textit{timesteps} em que há ocupação na zona térmica e a temperatura operativa ultrapassa o limite superior determinado pelo método adaptativo}%,
%user1= {-} %\unexpanded{\si{\celsius}}
}
\newglossaryentry{tsocup}{	% how the symbol will be called in the text \gls{x}
type=simbolos,		% set the glossary entry type as "symbol"
name={\ensuremath{timesteps_{ocup}}},	
description={Número de \textit{timesteps} em que há ocupação na zona térmica}%,
%user1= {-} %\unexpanded{\si{\celsius}}
}

\newglossaryentry{tsupv}{	% how the symbol will be called in the text \gls{x}
type=simbolos,		% set the glossary entry type as "symbol"
name={\ensuremath{T_{sup,v}}},	
description={Temperatura limite superior na faixa de conforto, considerando-se a velocidade do ar},
user1= \unexpanded{\si{\celsius}}
}
\newglossaryentry{tvar}{	% how the symbol will be called in the text \gls{x}
type=simbolos,		% set the glossary entry type as "symbol"
name={\ensuremath{T_{v_{ar}}}},	
description={Margem extra de temperatura permitida pela consideração da velocidade do ar},
user1= \unexpanded{\si{\celsius}}
}
\newglossaryentry{rmsecp}{	% how the symbol will be called in the text \gls{x}
type=simbolos,		% set the glossary entry type as "symbol"
name={\ensuremath{RMSE_{Cp}}},
description={RMSE das diferenças entre os valores dos Cp's obtidos pela base da TPU e obtidos pelo MA},
user1= {-}%\unexpanded{\si{\celsius}}
}
\newglossaryentry{alphaj}{	% how the symbol will be called in the text \gls{x}
type=simbolos,		% set the glossary entry type as "symbol"
name={\ensuremath{\alpha_j}},	
description={Ângulo de incidência do vento  sobre a fachada, e tem valor igual a $30 \cdot j$},
user1= \ensuremath{^{\circ}}
}
\newglossaryentry{meancptpu}{	% how the symbol will be called in the text \gls{x}
type=simbolos,		% set the glossary entry type as "symbol"
name={\ensuremath{Cp^{TPU}_{f_i,\alpha_j,p_k}}},	
description={Valor do Cp disponibilizado pela base de dados da TPU para a fachada $i$ de uma edificação, para o ângulo de incidência do vento igual a $\alpha_j$, no ponto $k$}, %\glsname{tpu} \glsname{alphaj}}
user1= {-}%\unexpanded{\si{\celsius}}
}
\newglossaryentry{cpma}{	% how the symbol will be called in the text \gls{x}
type=simbolos,		% set the glossary entry type as "symbol"
name={\ensuremath{Cp^{MA}_{f_i,\alpha_j}}},	
description={Cp calculado pelo MA para a fachada de uma edificação com proporções iguais às da fachada $i$, para o ângulo de incidência do vento igual a $\alpha_j$}, %\glsname{ma} \glsname{alphaj}
user1= {-}%\unexpanded{\si{\celsius}}
}
\newglossaryentry{fi}{	% how the symbol will be called in the text \gls{x}
type=simbolos,		% set the glossary entry type as "symbol"
name={\ensuremath{f_i}},	
description={Fachada $i$ da edificação avaliada},
user1= {-}%\unexpanded{\si{\celsius}}
}
\newglossaryentry{m}{	% how the symbol will be called in the text \gls{x}
type=simbolos,		% set the glossary entry type as "symbol"
name={\ensuremath{\dot{m}_{i,j}}},	
description={Fluxo de ar entre os pontos $i$ e $j$, quando a porta/janela está aberta},
user1=\unexpanded{\si{\kilogram/\second}}
}
\newglossaryentry{n}{	% how the symbol will be called in the text \gls{x}
type=simbolos,		% set the glossary entry type as "symbol"
name={\ensuremath{\dot{n}_{i,j}}},	
description={Fluxo de ar entre os pontos $i$ e $j$, quando a porta/janela está fechada},
user1=\unexpanded{\si{\kilogram/\second}}
}
\newglossaryentry{theta}{	% how the symbol will be called in the text \gls{x}
type=simbolos,		% set the glossary entry type as "symbol"
name={\ensuremath{\Theta}},	
description={Fração de abertura da porta/janela},
user1={-} %\unexpanded{\si{\kilogram/\second}}
}
\newglossaryentry{height}{	% how the symbol will be called in the text \gls{x}
type=simbolos,		% set the glossary entry type as "symbol"
name={\ensuremath{H}},	
description={Altura da abertura},
user1= \unexpanded{\si{\meter}}
}
\newglossaryentry{width}{	% how the symbol will be called in the text \gls{x}
type=simbolos,		% set the glossary entry type as "symbol"
name={\ensuremath{W}},	
description={Largura da abertura},
user1= \unexpanded{\si{\meter}}
}
\newglossaryentry{piz}{	% how the symbol will be called in the text \gls{x}
type=simbolos,		% set the glossary entry type as "symbol"
name={\ensuremath{P_{i(z)}}},	
description={Pressão de ar no ponto $i$, altura $z$},
user1= \unexpanded{\si{\pascal}}
}
\newglossaryentry{cqq}{	% how the symbol will be called in the text \gls{x}
type=simbolos,		% set the glossary entry type as "symbol"
name={\ensuremath{C_Q}},	
description={Coeficiente de fluxo mássico de ar},
user1={-} %\unexpanded{\si{\kilogram/\second}}
}
\newglossaryentry{cdd}{	% how the symbol will be called in the text \gls{x}
type=simbolos,		% set the glossary entry type as "symbol"
name={\ensuremath{C_d}},	
description={Coeficiente de descarga da abertura},
user1={-} %\unexpanded{\si{\kilogram/\second}}
}
\newglossaryentry{cpp}{	% how the symbol will be called in the text \gls{x}
type=simbolos,		% set the glossary entry type as "symbol"
name={\ensuremath{C_p}},	
description={Coeficiente de pressão},
user1={-} %\unexpanded{\si{\kilogram/\second}}
}
\newglossaryentry{cppeq}{	% how the symbol will be called in the text \gls{x}
type=simbolos,		% set the glossary entry type as "symbol"
name={\ensuremath{C{_p,eq}}},	
description={Coeficiente de pressão equivalente},
user1={-} %\unexpanded{\si{\kilogram/\second}}
}
\newglossaryentry{exp}{	% how the symbol will be called in the text \gls{x}
type=simbolos,		% set the glossary entry type as "symbol"
name={\ensuremath{exp}},	
description={Expoente de fluxo de massa de ar},
user1={-} %\unexpanded{\si{\kilogram/\second}}
}
\newglossaryentry{pzn}{	% how the symbol will be called in the text \gls{x}
type=simbolos,		% set the glossary entry type as "symbol"
name={\ensuremath{P_{zn}}},	
description={Pressão do ar na zona térmica analisada},
user1=\unexpanded{\si{\pascal}}
}
\newglossaryentry{pi}{	% how the symbol will be called in the text \gls{x}
type=simbolos,		% set the glossary entry type as "symbol"
name={\ensuremath{P_i}},	
description={Pressão do ar na zona térmica, ligada pela porta $i$},
user1=\unexpanded{\si{\pascal}}
}
\newglossaryentry{np}{	% how the symbol will be called in the text \gls{x}
type=simbolos,		% set the glossary entry type as "symbol"
name={\ensuremath{N_P}},	
description={Número de portas que se conectam à zona térmica}
}
\newglossaryentry{nd}{	% how the symbol will be called in the text \gls{x}
type=simbolos,		% set the glossary entry type as "symbol"
name={\ensuremath{N_d}},	
description={Número de pontos de $C_p$ disponibilizados na fachada do edifício}
}
\newglossaryentry{nj}{	% how the symbol will be called in the text \gls{x}
type=simbolos,		% set the glossary entry type as "symbol"
name={\ensuremath{N_J}},	
description={Número de janelas que se conectam à zona térmica}
}
\newglossaryentry{li}{	% how the symbol will be called in the text \gls{x}
type=simbolos,		% set the glossary entry type as "symbol"
name={\ensuremath{L_i}},	
description={Perímetro da porta $i$},
user1= \unexpanded{\si{\meter}}
}
\newglossaryentry{area}{	% how the symbol will be called in the text \gls{x}
type=simbolos,		% set the glossary entry type as "symbol"
name={\ensuremath{A_j}},	
description={Área da janela $j$},
user1= \unexpanded{\si{\square\meter}}
}
\newglossaryentry{pext}{	% how the symbol will be called in the text \gls{x}
type=simbolos,		% set the glossary entry type as "symbol"
name={\ensuremath{P_{ext}}},	
description={Pressão do ar no ambiente externo},
user1=\unexpanded{\si{\pascal}}
}
\newglossaryentry{cpj}{	% how the symbol will be called in the text \gls{x}
type=simbolos,		% set the glossary entry type as "symbol"
name={\ensuremath{C_{p,j}}},	
description={Cp na superície da janela $j$},
user1={-} %\unexpanded{\si{\kilogram/\second}}
}
\newglossaryentry{deltamet}{	% how the symbol will be called in the text \gls{x}
type=simbolos,		% set the glossary entry type as "symbol"
name={\ensuremath{\delta_{met}}},	
description={Espessura da camada limite para o perfil de vento na estação meteorológica},
user1=\unexpanded{\si{\meter}}
}
\newglossaryentry{alphamet}{	% how the symbol will be called in the text \gls{x}
type=simbolos,		% set the glossary entry type as "symbol"
name={\ensuremath{\alpha_{met}}},	
description={Expoente para o perfil de vento na estação meteorológica},
user1={-} %\unexpanded{\si{\kilogram/\second}}
}