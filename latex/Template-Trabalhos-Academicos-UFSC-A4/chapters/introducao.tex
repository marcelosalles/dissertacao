% ----------------------------------------------------------
\chapter{Introdução}
\label{chapter:introducao}

De acordo com a \acrlong{iea} \cite{IEA2018a}, no ano de 2017 o setor de edificações representou mais de 30\% do consumo final total de energia no mundo. 
O relatória da \citeonline{IEA2018} aponta que a demanda por energia destinada ao resfriamento de ar em edificações mais que triplicou do ano de 1990 a 2016 e, se não houver mudanças no cenário atual, estima-se que essa demanda mais que triplicará até o ano de 2050, representando 37\% do aumento no consumo de eletricidade em edificações. Isso corresponderá a 11,5\% do consumo de energia total em edificações comerciais. 
O potencial de aumento na demanda por energia destinada ao resfriamento de ar em países de clima quente é ainda mais expressivo. Das 2,8 bilhões de pessoas que vivem nas partes mais quentes do mundo hoje, apenas 8\% possuem sistema de condicionamento de ar \cite{IEA2018}. No Brasil, a parcela do resfriamento de ar nas cargas de pico das redes elétricas em 2016 correspondia a 7,6\% do total, e a estimativa, considerando-se o cenário base, é de que essa parcela represente 30,8\% da carga de pico até o ano de 2050 \cite{IEA2018}. 
%No Brasil, desde 2009 o INMETRO possui um programa de etiquetagem de edificações voltado para padrões de eficiência energética de edificações \cite{BRASIL2009}.

Esse contexto evidencia a necessidade de medidas mitigadoras relacionadas ao consumo energético destinado ao resfriamento de edificações. O uso de técnicas de resfriamento passivo, como a \acrfull{vn}, pode ser uma solução. O resfriamento passivo é um conjunto de técnicas sustentáveis para resfriar edifícios por meios naturais \cite{Samani2016}, que consiste em qualquer sistema que busca minimizar, ou eliminar, se possível, o uso de sistemas de condicionamento de ar, com o objetivo de reduzir as altas temperaturas internas e o consumo de energia para resfriamento, proporcionando conforto térmico aos ocupantes.

Técnicas de \acrshort{vn} são encontradas ao longo de toda a história na arquitetura vernacular \cite{Pesic2018}, e hoje vêm sendo atualizadas de acordo com novos estudos no campo de conforto térmico e projetos sustentáveis de edificações.
Além de assegurar a qualidade do ar, a \acrshort{vn} promove o resfriamento da edificação, proporcionando conforto térmico aos usuários quando as condições do clima externo são favoráveis \cite{Yao2009}.

Apesar de um quinto da energia elétrica brasileira ser destinada a edificações comerciais, de serviços e públicas \cite{EPE2018}, o uso de \acrshort{vn} em edifícios de escritórios vem diminuindo gradualmente no Brasil. De acordo com \citeonline{Alves2017}, apesar do uso de sistemas de condicionamento de ar e iluminação mais eficientes, edifícios da cidade de Belo Horizonte construídos a partir dos anos 2000 tendem a ser os maiores consumidores de energia, por não adotarem estratégias de resfriamento passivo.

Para que o conforto térmico dos usuários seja garantido sem um consumo significativo de energia, é importante entender como ocorrem as variações térmicas em um edifício antes de construí-lo. Análises durante os estágios iniciais de projeto de uma edificação com \acrshort{vn} apontam decisões fundamentais para o desempenho térmico. No estágio inicial de projeto, o potencial de otimização é significativo, e nesta etapa qualquer estimativa do conforto e desempenho energético da edificação pode refletir nas tomadas de decisão \cite{Belleri2014, Roetzel2014}.

O método mais avançado de se estimar o desempenho termoenergético de edificações atualmente é por meio de simulações computacionais. No entanto, esse processo exige o conhecimento técnico de um especialista, pois simulações termoenergéticas dinâmicas requerem modelos detalhados e enfrentam diversos problemas, associados principalmente a informações necessárias para dados de entrada do modelo processado \cite{Corgnati2013}. No contexto brasileiro, a análise do desempenho térmico de edificações por meio de simulações computacionais é uma medida relevante, pois, assim como em outros países em desenvolvimento, a falta de acessibilidade a dados relacionados a padrões de consumo de energia e atributos físicos e operacionais de edifícios de escritório dificulta as análises a partir de bancos de dados \cite{Alves2018}. Uma alternativa para contornar essas questões é o desenvolvimento de modelos a partir de simulações computacionais, os metamodelos. Por meio de metamodelos é possível se obter resultados próximos aos de simulações complexas de desempenho energético.

Metamodelos para eficiência energética de edificações podem ser desenvolvidos a partir de diferentes métodos \cite{Ostergard2018}. A solução mais apropriada depende do contexto e propósitos de cada aplicação.
\citeonline{Versage2015} foi capaz de estimar as cargas térmicas de edificações comerciais através de diferentes métodos de metamodelagem.
\citeonline{Melo2016} desenvolveram um modelo de \acrfull{ann} para estimar graus hora de resfriamento e cargas térmicas de aquecimento e resfriamento em edificações residenciais.
O desenvolvimento de um metamodelo de máquina de vetores de suporte capaz de estimar conforto térmico em edificações comerciais foi proposto por \citeonline{Rackes2016}. Voltado principalmente a tipologias de escolas, o metamodelo estima a fração de horas em desconforto por calor dos ocupantes ao longo do ano.
%
%A \acrshort{vn} em edificações apresenta comportamentos complexos e a avaliação do seu potencial de resfriamento faz-se necessária desde a fase inicial de projeto. Possibilitar esse tipo de análise de forma simples e rápida é fundamental nas tomadas de decisão em projetos de edificações e na aplicação de políticas públicas voltadas à eficiência energética. Por meio de ferramentas de aprendizagem automática, surge a oportunidade de se desenvolver metamodelos capazes de obter resultados de conforto térmico em edificações.

O consumo de energia para resfriamento de edificações é expressivo no mundo, e a expectativa é de que a demanda por energia continue crescendo nas próximas décadas, principalmente em países de climas quentes.
Neste contexto, o uso de técnicas de \acrfull{vn} apresenta-se como uma solução para mitigar o uso de condicionamento de ar.
Entretanto, o desempenho térmico das edificações apresenta fenômenos termofísicos complexos, fazendo com que estimativas de conforto térmico devam ser consideradas preferencialmente desde as etapas iniciais de projeto. 	
Na busca por uma ferramenta capaz de auxiliar projetistas de maneira rápida e simples, surge a possibilidade de utilizar-se metamodelos.
Portanto, este trabalho apresenta o desenvolvimento de um metamodelo capaz de estimar o conforto térmico em edifícios de escritórios ventilados naturalmente.

\section{Objetivos}
\subsection{Objetivo geral}

O objetivo deste estudo é desenvolver um metamodelo capaz de estimar o conforto térmico em edifícios de escritórios ventilados naturalmente.

\subsection{Objetivos específicos}

Dentre os objetivos específicos deste trabalho, destacam-se:

\begin{itemize}
	\item Identificar as características construtivas encontradas em edifícios de escritórios ventilados naturalmente na cidade de São Paulo;
	\item Desenvolver um modelo de simulação termoenergética simplificado, com apenas uma zona térmica, capaz de representar as trocas térmicas de uma sala em um edifício de escritórios;
	\item Definir as variáveis com maior e menor influência no desempenho térmico dos edifícios ventilados naturalmente.
\end{itemize}